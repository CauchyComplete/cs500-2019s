\documentclass[11pt,a4paper,oneside,microtype,nokorean]{oblivoir}
\usepackage[margin=1in]{geometry}
\usepackage[utf8]{inputenc}
\usepackage[english]{babel}
\usepackage{amsmath}
\usepackage{amsthm}
\usepackage{amssymb}
\usepackage{hyperref}
\usepackage{cleveref}
\usepackage{algorithm}
\usepackage{algpseudocode}

\newtheorem{example}{Example}
\newtheorem{theorem}{Theorem}
\newtheorem{lemma}[theorem]{Lemma}

\newcommand{\kwtrue}{\textsl{true}}
\newcommand{\kwfalse}{\textsl{false}}

% New definitions
\algnewcommand\algorithmicswitch{\textbf{switch}}
\algnewcommand\algorithmiccase{\textbf{case}}
\algnewcommand\algorithmicassert{\texttt{assert}}
\algnewcommand\Assert[1]{\State \algorithmicassert(#1)}%
% New "environments"
\algdef{SE}[SWITCH]{Switch}{EndSwitch}[1]{\algorithmicswitch\ #1\ \algorithmicdo}{\algorithmicend\ \algorithmicswitch}%
\algdef{SE}[CASE]{Case}{EndCase}[1]{\algorithmiccase\ #1}{\algorithmicend\ \algorithmiccase}%
\algtext*{EndSwitch}%
\algtext*{EndCase}%


\begin{document}

\title{Lecture Note on Search Algorithms}
\author{Jeehoon Kang}
\maketitle

\section{Breadth-First Search (BFS)}

\paragraph{Algorithm}

Here is the BFS algorithm that finds all reachable vertices from a given vertex:

\begin{algorithm}
  \caption{Breadth-First Search Algorithm}\label{bfs}
  \begin{algorithmic}[1]
    \Procedure{BFS}{$G,v_i$} \Comment{BFS search algorithm}
    \State $\Call{BFSInner}{G,\emptyset,\{v_i\}}$
    \EndProcedure
    \Statex
    \Procedure{BFSInner}{$G=(V,E),X,F$} \Comment{A helper function for BFS}
    \If{$F = \emptyset$} \Comment{(needs: set empty predicate)}
    \State \textbf{return} $X$
    \EndIf
    \State $X' = X \cup F$ \Comment{(needs: set union)}
    \State $F' = N^+_G(F) \setminus X'$ \Comment{(needs: set flatten, set minus)}
    \State $\Call{BFSInner}{G,X',F'}$
    \EndProcedure
  \end{algorithmic}
\end{algorithm}

The above algorithm omits to implement the set operations highlighted above.  We present two
strategies for implementing sets: representing them as a boolean sequence, or as an element
sequence.  We begin with representing both $X$ and $F$ as boolean sequences.


\paragraph{Representing Set as a Boolean Sequence}

Let $V = \{v_0,v_1,\cdots,v_{|V|-1}\}$.  Then the sets $X, F \subseteq V$ can be represented as a
sequence of booleans of length $|V|$, where a sequence $S \in \mathbb{S}_{\mathbb{B}}$ represents
the set consisting of those elements $v_i$ for which $S_i$ is true.  For example, if
$V = \{v_0,v_1,v_2\}$, then $\langle \kwfalse, \kwtrue, \kwtrue \rangle$ represents the set
$\{v_1,v_2\}$.

Set operations are defined as follows:

\begin{itemize}
\item $\texttt{empty} = \texttt{tabulate}~(\lambda \_.~\kwfalse)~|V|$
\item $\texttt{is\_empty}(X) = \neg (\texttt{reduce}~\lor~\kwfalse~X)$
\item $\texttt{union}~A~B = \texttt{combine}~\lor~A~B$
\item $\textrm{minus}~A~B = \texttt{combine}~(\lambda a~b.~a \land \neg b)~A~B$
\item $\texttt{flatten}~\mathcal{A} = \texttt{reduce}~\texttt{union}~\texttt{empty}~\mathcal{A}$
\end{itemize}

By plugging in the above set operations, we have a complete BFS implementation.  It has $O(|V|^2)$
work and $O(|V|~lg~|V|)$ span.


\paragraph{Representing $F$ as an Index Sequence}

Representing $F$ as a boolean sequence is wasteful because, regardless of how many elements are in
$F$, it will iterate over all $|V|$ vertices at line 9, which dominates the work of the entire
algorithm.

To optimize line 9, we represent $F$ as a sequence of indexes instead of as a sequence of booleans.
The set $F \subseteq V$ can be represented as a sequence of indexes whose corresponding elements are
in $F$.  For example, if $V = \{v_0,v_1,v_2\}$, then both $\langle 1,2 \rangle$ and
$\langle 2,1 \rangle$ represent the set $\{v_1,v_2\}$.

Set operations are defined as follows (here, $F,A,B$ are represented as an index sequence and $X$ as
a boolean sequence):

\begin{itemize}
\item $\texttt{convert}~X = \texttt{map}~\texttt{fst}~(\texttt{filter}~\texttt{snd}~(\texttt{mapi}~(\lambda i,b. (i,b))~X))$
\item $\texttt{convert}~F = (\texttt{let}~F' = \texttt{sort}~F~\texttt{in}~\cdots)$
\item $\texttt{empty} = \langle~\rangle$
\item $\texttt{is\_empty}(X) = (|X| = 0)$
\item $\texttt{union}~A~B = \texttt{append}~A~B$
\item $\textrm{minus}~F~X = \texttt{filter}~(\lambda f.~\neg X_f)~F$
\item $\texttt{flatten}~\mathcal{A} = \texttt{reduce}~\texttt{union}~\texttt{empty}~\mathcal{A}$
\end{itemize}

By plugging in the above set operations, we have another complete BFS implementation, whose work is
$O(|E|)$ and span is $O(d~lg~|V|)$, where $d$ is the maximum length from $v_i$ in $G$.


\paragraph{Correctness} The following lemma on \textsc{BFSInner} is the key for BFS's correctness:

\begin{lemma}[Inductive invariant of \textsc{BFSInner}] In the recursion tree of
  $\textsc{BFS}(G,v_i)$, for each invocation $\textsc{BFSInner}(G,X,F)$, the following condition
  holds:
  \[ \forall v.~\mbox{($v$ is reachable from $v_i$ in $G$)} \iff (v \in X~\lor~\mbox{($v$ is
      reachable from $F$ in $G \setminus X$)}).
  \]  
\end{lemma}
\begin{proof}
  By induction.  For the base case, the first invocation $\textsc{BFSInner}(G,\emptyset,\{v_i\})$
  trivially satisfies the above condition.  For the inductive case, consider the invocation
  $\textsc{BFSInner}(G,X',F')$ from another one $\textsc{BFSInner}(G,X,F)$.

  The remainder of the proof is left as homework.  (Question: how did we find the inductive
  invariant?)
\end{proof}

From the above lemma follows the correctness of the algorithm:

\begin{theorem}[Correctness of \textsc{BFS}] For any $G$, $v_i$, and $v$, $v$ is reachable from
  $v_i$ in $G$ if and only if $v \in \textsc{BFS}(G,v_i)$.
\end{theorem}



\section{Depth-First Search (DFS)}

\paragraph{Algorithm}

Here is the DFS algorithm that finds all reachable vertices from a given vertex:

\begin{algorithm}
  \caption{Depth-First Search Algorithm}\label{dfs}
  \begin{algorithmic}[1]
    \Procedure{DFS}{$G,v_i$} \Comment{DFS search algorithm}
    \State $\Call{DFSInner}{G,\emptyset,\langle v_i\rangle}$
    \EndProcedure
    \Statex
    \Procedure{DFSInner}{$G=(V,E),X,S$} \Comment{A helper function for DFS}
    \Switch {$\textrm{splitHead}(S)$}
    \Case {\textrm{None}}
    \State $X$
    \EndCase
    \Case {$\textrm{Some}(v,S_t)$}
    \State $X' = X \cup \{v\}$
    \State $S' = (v \in X~\texttt{?}~\langle \rangle~\texttt{:}~N^+_G(v))~\texttt{++}~S_t$
    \State $\Call{DFSInner}{G,X',S'}$
    \EndCase
    \EndSwitch
    \EndProcedure
  \end{algorithmic}
\end{algorithm}

The sequence $S$ (representing a stack) is implemented with a linked list.


\paragraph{Cost}

In the recursion tree of \textsc{DFS}, there are at most $|E|$ invocations of \textsc{DFSInner}, and
the vertices in total are inserted into the stack ($S$) at most $|E|$ times.  Thus \textsc{DFS} has
$O(|E|)$ work.  Since the algorithm is completely sequential, it has $O(|E|)$ span.



\paragraph{Correctness} The correctness proof of DFS is similar to that of BFS.



\section{Dijkstra's Algorithm}

\paragraph{Algorithm}

Here is the Dijkstra's algorithm that finds the shortest path weight from a given vertex to all
vertices, assuming that all edge weights are non-negative.


\begin{algorithm}
  \caption{Dijkstra's Algorithm}\label{dijkstra}
  \begin{algorithmic}[1]
    \Procedure{Dijkstra}{$G,v_i$} \Comment{Dijkstra's shortest path algorithm}
    \State $D_0 = \langle \infty, \cdots, \infty, 0, \infty, \cdots, \infty \rangle$ \Comment{$\infty$ except for the initial vertex}
    \State $\Call{DijkstraInner}{G,D_0,\emptyset,\{v_i\}}$
    \EndProcedure
    \Statex
    \Procedure{DijkstraInner}{$G=(V,E),D,X,F$} \Comment{A helper function for Dijkstra}
    \If{$F = \emptyset$}
    \State \textbf{return} $X$
    \EndIf
    \State Let $f_0 \in F$ be a frontier with the minimum distance $D_{f_0}$
    \State $D' = \texttt{combine}~\textrm{min}~D~(N^+_G(f_0) + D_{f_0})$ \Comment{(abusing a lot of notations)}
    \State $X' = X \cup \{f_0\}$
    \State $F' = (F \cup N^+_G(f_0)) \setminus X'$
    \State $\Call{DijkstraInner}{G,D',X',F'}$
    \EndProcedure
  \end{algorithmic}
\end{algorithm}


\paragraph{Cost}

When representing $X$ as a boolean sequence and $F$ as an index sequence, Dijkstra's algorithm's
work is $O(|V|^2)$ and span is $O(|V|~lg~|V|)$.  By using a more sophisticated data structure
(e.g. the priority queue), we can optimize (work-wise) Dijkstra's algorithm.



\paragraph{Correctness} The following lemma on \textsc{DijkstraInner} is the key for Dijkstra's
correctness:

\begin{lemma}[Inductive invariant of \textsc{DijkstraInner}] Suppose $\delta_G(v,w)$ is the shortest
  path weight from $v$ to $w$ in $G$.  In the recursion tree of $\textsc{Dijkstra}(G,v_i)$, for each
  invocation $\textsc{DijkstraInner}(G,D,X,F)$, the following conditions holds:
  \begin{itemize}
  \item $X \cap F = \emptyset$;
  \item $\forall v \notin X \cup F, D_v = \infty$;
  \item $\forall v\in X, w \notin X.~D_v \le D_w$;
  \item $\forall v \in X.~\delta_G(v_i,v) = D_v$; and
  \item
    $\forall v \notin X.~\delta_G(v_i,v) = \textrm{min}_{f \in F} (D_f + (\delta_{G \setminus
      X}(f,v)))$.
  \end{itemize}

  (Question: how did we find the inductive invariant?)
\end{lemma}
\begin{proof}
  By induction.

  \begin{itemize}
  \item In the initial case, we have $(D, X, F) = (D_0, \emptyset, \{v_i\})$.  All conditions trivially hold.
  \item In the inductive case, we have:
    \begin{align*}
      \forall v.~D'_v =&~ \textrm{min}(D_v, w_G(f_0,v)) \\
      X' =&~ X \cup F \\
      F' =&~ (F \cup N^+_G(f_0)) \setminus X'~,
    \end{align*}
    where $D', X', F'$ are the arguments for the recursive invocation of \textsc{DijkstraInner}.
    Now we prove that the conditions hold for $D', X', F'$, assuming they hold for $D, X, F$ as the
    inductive hypothesis.

    \begin{itemize}
    \item $X' \cap F' = X' \cap ((F \cup N^+_G(f_0)) \setminus X') = \emptyset$.

    \item If $v \notin X' \cup F'$, we have $v \notin X \cup F \cup N^+_G(f_0)$ and thus
      $v \notin X \cup F$ and $v \notin N^+_G(f_0)$.  Then we have:
      %
      \begin{align*}
        D'_v
        =&~ \textrm{min}(D_v, w_G(f_0,v)) \\
        =&~ \textrm{min}(\infty, w_G(f_0,v)) & \mbox{(by $v \notin X \cup F$ and IH)} \\
        =&~ \textrm{min}(\infty, \infty) & \mbox{(by $v \notin N^+_G(f_0)$)} \\
        =&~ \infty.
      \end{align*}

    \item If $v \in X'$, then $D'_v \le D_{f_0}$.  If $w \notin X'$, then $D'_w \ge D_{f_0}$.  Thus
      $\forall v\in X, w \notin X.~D_v \le D_w$.

    \item If $v \in X'$, then either $v \in X$ or $v = f_0$.

      If $v \in X'$, then $\delta_G(v_i,v) = D_v$ from the IH's fourth condition, and $D_v = D'_v$
      from the IH's third condition.

      If $v = f_0$, we have:
      \begin{align*}
        \delta_G(v_i,v) =&~ \textrm{min}_{f \in F} (D_f + (\delta_{G \setminus X}(f,v))) & \mbox{(by the IH's
                                                                                           fifth condition)} \\
        =&~ D_{f_0} & \mbox{($\because [\forall f \in F.~D_{f_0} \le D_f]$ and
                      $\delta_{G \setminus X}(f_0,v) = 0$)} \\
        =&~ D'_{f_0}~. & \mbox{($\because 0 \le w_G(f_0,f_0)$)}
      \end{align*}

    \item If $v \notin X'$, then we have $v \notin X$ and $v \neq f_0$.  We have:
      \begin{align*}
        \delta_G(v_i,v) =&~ \textrm{min}_{f \in F} (D_f + (\delta_{G \setminus X}(f,v))) & \mbox{(by the IH's
                                                                                           fifth cond.)} \\
        =&~ \textrm{min} [ D_{f_0} + (\delta_{G \setminus X}(f_0,v)), \\
                         &~ ~~~~~~\textrm{min}_{f \in F \setminus \{f_0\}} (D_f + \delta_{G \setminus X}(f,v)) ] \\
        =&~ \textrm{min} [ D_{f_0} + \textrm{min}_{u \in N^+_G(f_0) \setminus X'}(w_G(f_0,u) + \delta_{G \setminus X'}(u,v)), & \mbox{($\because f_0 \neq v$)} \\
                         &~ ~~~~~~\textrm{min}_{f \in F \setminus \{f_0\}} (D_f + \delta_{G \setminus X}(f,v)) ] \\
        =&~ \textrm{min} [ D_{f_0} + \textrm{min}_{u \in N^+_G(f_0) \setminus X'}(w_G(f_0,u) + \delta_{G \setminus X'}(u,v)), & \mbox{($\because$ visiting $f_0$ is bad} \\
                         &~ ~~~~~~\textrm{min}_{f \in F \setminus \{f_0\}} (D_f + \delta_{G \setminus X'}(f,v)) ] & \mbox{as $D_{f_0} \le D_f$)} \\
        =&~ \textrm{min}_{f \in (N^+_G(f_0) \setminus X') \cup (F \setminus \{f_0\})} \\
        &~ ~~~~~~[\textrm{min}(D_{f_0} + w_G(f_0,f), D_f) + \delta_{G \setminus X'}(f,v)] \\
        =&~ \textrm{min}_{f \in F'} D'_f + \delta_{G \setminus X'}(f,v)
      \end{align*}
    \end{itemize}
  \end{itemize}
\end{proof}


From the above lemma follows the correctness of the algorithm:

\begin{theorem}[Correctness of \textsc{Dijkstra}] For any $G$, $v_i$, and $v$,
  $\textsc{Dijkstra}(G,v_i)_v$ is the shortest path weight from $v_i$ to $v$.
\end{theorem}



\section{Bellman-Ford's Algorithm}

\paragraph{Algorithm}

Here is the Bellman-Ford's algorithm that finds the shortest path weight from a given vertex to all
vertices, assuming that there are possibly edges with negative weight but no cycles of negative
weight.

\begin{algorithm}
  \caption{Bellman-Ford's Algorithm}\label{dijkstra}
  \begin{algorithmic}[1]
    \Procedure{Bellman-Ford}{$G,v_i$} \Comment{Bellman-Ford's shortest path algorithm}
    \State $D_0 = \langle \infty, \cdots, \infty, 0, \infty, \cdots, \infty \rangle$ \Comment{$\infty$ except for the initial vertex}
    \State $\Call{Bellman-FordInner}{G,D_0,0}$
    \EndProcedure
    \Statex
    \Procedure{Bellman-FordInner}{$G=(V,E),D,k$} \Comment{A helper function}
    \If{$k = |V|$}
    \State \textbf{return} $D$
    \EndIf
    \State Let $D' = [i \mapsto \textrm{min}(D_i, \textrm{min}_{v_j \in N^-_G(v_i)} (D_j + w_G(i,j)))]$
    \State $\Call{Bellman-FordInner}{G,D',k+1}$
    \EndProcedure
  \end{algorithmic}
\end{algorithm}


\paragraph{Cost}

In the recursion tree, an invocation of \textsc{Bellman-FordInner} has $O(|E|)$ work and $O(lg~|V|)$
span.  Thus the total work and span of \textsc{Bellman-Ford} is $O(|V||E|)$ and $O(|V|~lg~|V|)$.



\paragraph{Correctness} The following lemma on \textsc{Bellman-FordInner} is the key for Bellman-Ford's
correctness:

\begin{lemma}[Inductive invariant of \textsc{Bellman-FordInner}] In the recursion tree of
  $\textsc{Bellman-Ford}(G)$, for each invocation $\textsc{Bellman-FordInner}(G,D,k)$, the
  following condition holds:
  \[ \forall i,j.~D_{i,j} = \delta_G^k(v_i,v_j),
  \]  
  where $\delta_G^k(v_i,v_j)$ is the shortest weight of the paths from $v_i$ to $v_j$ in $G$ of
  length $\le k+1$.
\end{lemma}
\begin{proof}
  Homework.  (Question: how did we find the inductive invariant?)
\end{proof}

From the above lemma follows the correctness of the algorithm:

\begin{theorem}[Correctness of \textsc{Bellman-Ford}] For any $G$, $i$, and $j$,
  $\textsc{Bellman-Ford}(G,v_i)_{j}$ is the shortest path weight from $v_i$ to $v_j$.
\end{theorem}



\section{Floyd-Warshall's Algorithm}

\paragraph{Algorithm}

Here is the Floyd-Warshall's algorithm that finds the shortest path weight from all vertices to all
vertices, assuming that there are possibly edges with negative weight but no cycles of negative
weight.

\begin{algorithm}
  \caption{Floyd-Warshall's Algorithm}\label{dijkstra}
  \begin{algorithmic}[1]
    \Procedure{Floyd-Warshall}{$G = (V,E)$} \Comment{Floyd-Warshall's shortest path algorithm}
    \State $D_0 = E$ \Comment{The initial distance matrix}
    \State $\Call{Floyd-WarshallInner}{G,D_0,0}$
    \EndProcedure
    \Statex
    \Procedure{Floyd-WarshallInner}{$G=(V,E),D,k$} \Comment{A helper function}
    \If{$k = |V|$}
    \State \textbf{return} $D$
    \EndIf
    \State Let $D' = [(i,j) \mapsto \textrm{min}(D_{i,j}, D_{i,k} + D_{k,j})]$
    \State $\Call{Floyd-WarshallInner}{G,D',k+1}$
    \EndProcedure
  \end{algorithmic}
\end{algorithm}


\paragraph{Cost}

In the recursion tree, an invocation of \textsc{Floyd-WarshallInner} has $O(|V|^2)$ work and $O(1)$
span.  Thus the total work and span of \textsc{Floyd-Warshall} is $O(|V|^3)$ and $O(|V|)$.


\paragraph{Correctness} The following lemma on \textsc{Floyd-WarshallInner} is the key for
Floyd-Warshall's algorithm's correctness:

\begin{lemma}[Inductive invariant of \textsc{Floyd-WarshallInner}] In the recursion tree of
  $\textsc{Floyd-Warshall}(G)$, for each invocation $\textsc{Floyd-WarshallInner}(G,D,k)$, the
  following condition holds:
  \[ \forall i,j.~D_{i,j} = \delta_G^k(v_i,v_j),
  \]
  where $\delta_G^k(v_i,v_j)$ is the shortest weight of the paths from $v_i$ to $v_j$ in $G$ where
  the intermediate vertices are in $\{0,1,\cdots,(k-1)\}$.
\end{lemma}
\begin{proof}
  Homework.  (Question: how did we find the inductive invariant?)
\end{proof}

From the above lemma follows the correctness of the algorithm:

\begin{theorem}[Correctness of \textsc{Floyd-Warshall}] For any $G$, $i$, and $j$,
  $\textsc{Floyd-Warshall}(G)_{i,j}$ is the shortest path weight from $v_i$ to $v_j$.
\end{theorem}



\end{document}
