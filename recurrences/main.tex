\documentclass[11pt,a4paper,oneside,microtype,nokorean]{oblivoir}
\usepackage[margin=1in]{geometry}
\usepackage[utf8]{inputenc}
\usepackage[english]{babel}
\usepackage{amsmath}
\usepackage{amsthm}
\usepackage{hyperref}
\usepackage{cleveref}

\newtheorem{example}{Example}

\begin{document}

\title{Lecture Note on Recurrences}
\author{Jeehoon Kang}
\maketitle

\section{Introduction}

\paragraph{Study guide}

I recommend you to study recurrences like this:

\begin{itemize}
\item Read the textbook's Section 11 (pages 87-98).
\item Read \Cref{sec:examples} to see how proofs work.
\item Solve the examples and exercises in the textbook's Sections 11.4, 11.5 (pages 91-98) on
  yourselves.
\end{itemize}


\paragraph{Exam guide}

Here are a few tips for the quizzes and exams.

\begin{itemize}
\item The most foundational way to solve a recurrence is by induction (what is called ``substitution
  method'' in the textbook).  Even if someone is very pedantic, s/he should accept proofs by
  induction as far as s/he believes in Mathematics.

\item ``Tree method'' and ``brick method'' are lemmas that deal with all the difficulties of proof
  by induction.  In the exams, you can freely use these methods as lemmas, except when the problem
  specifically asks you to prove it by induction.

\item Tree method: analyzing the recursion tree, and then deriving a closed-form solution.

\item Brick method says:
  \begin{itemize}
  \item If a recursion tree is root-dominated, then the total work is asymptotically equivalent to
    the work at the root.
  \item If a recursion tree is leaf-dominated, then the total work is asymptotically equivalent to
    the number of leaves.
  \end{itemize}

\item You can assume the grader already knows what ``recursion tree'' and ``level'' means, and
  assume the root is at the level $0$ (unless specified otherwise).  But you should not assume the
  grader is aware of \emph{e.g.} what $C(v)$ or $D(v)$ means for recursion trees.
\end{itemize}


\section{Examples}
\label{sec:examples}


\begin{example}[Balanced] Suppose $W(n) = 2W(n/2) + O(n)$.  Prove $W(n) = O(n~lg~n)$.
\end{example}

\begin{proof}
  We prove using the tree method, \emph{i.e.} analyzing the recursion tree and deriving a
  closed-form solution (Example 11.4).

  \begin{itemize}
  \item By definition, there exists $c_1,c_2$ such that $W(n) \le 2W(n/2) + c_1 n + c_2.$
  \item (Draw the recursion tree.)
  \item We can deduce from the tree that for the level $i$, there are $2^i$ nodes and the cost of
    each node is $c_1 (n (1/2)^i) + c_2$.  (To be pedantic, we can prove it by induction.)
  \item Thus the cost of the level $i$ is $2^i \times (c_1 (n (1/2)^i) + c_2) = c_1 n + 2^i c_2$.

  \item Since the recursion tree has $lg~n + 1$ levels, we have:
    \begin{align*}
      W(n)
      = &~ \sum_{i=0}^{lg~n} (c_1 n + 2^i c_2) & \mbox{($i=0$ for the root, $i=lg~n$ for the
                                                 leaves)} \\
      \le &~ c_1 n (lg~n + 1) + 2^{lg~n + 1} c_2 \\
      = &~ c_1 n (lg~n + 1) + 2 n c_2 \\
      = &~ O(n~lg~n).
    \end{align*}
  \end{itemize}
\end{proof}


\begin{example}[Root-dominated] Suppose $W(n) = 2W(n/2) + O(n^2)$.  Prove $W(n) = O(n^2)$.
\end{example}

\begin{proof}
  Prove by exploiting the fact that $W(n)$ is root-dominated (Example 11.5).

  \begin{itemize}
  \item By definition, there exists $c_1,c_2$ such that $W(n) \le 2W(n/2) + c_1 n^2 + c_2.$
  \item Let $C(v)$ be the cost of node $v$ in the \emph{recursion tree}.  If $v$ is a node of size
    $n$, then $C(v) = c_1 n^2 + c_2$, and
    $\sum_{u \in D(v)} C(u) = 2(c_1 (n/2)^2 + c_2) = 2 c_1 (n/2)^2 + 2 c_2$, where by $D(v)$ we mean
    the set of children of $v$ in the recursion tree.

    (Caution: we assume the grader already knows what ``recursion tree'' means, but you should
    define $C(v)$ and $D(v)$ on yourselves.)
  \item We have:
    \begin{align*}
      \dfrac{\sum_{u \in D(v)} C(u)}{C(v)}
      = &~ \dfrac{2 c_1 (n/2)^2 + 2 c_2}{c_1 n^2 + c_2} \\
      = &~ \dfrac{1 + (4 c_2 / c_1) / n^2}{2 + (2 c_2 / c_1) / n^2} & \mbox{(by dividing by $c_1 n^2 / 2$)} \\
      \le &~ 3/4. & \mbox{(if $(4 c_2 / c_1) / n^2 \le 1/2$)}
    \end{align*}
  \item So if $(4 c_2 / c_1) / n^2 \le 1/2$, or $\sqrt{8 c_2 / c_1} \le n$, then when walking down
    the tree, the cost of a level in the recursion tree decreases more than a factor of $4/3$.
  \item Thus the root dominates the cost, and $W(n) = O(C(n)) = O(c_1 n^2 + c_2) = O(n^2)$.
  \end{itemize}
\end{proof}


\begin{example}[Leaf-dominated] Suppose $W(n) = 2W(n/2) + O(\sqrt{n})$.  Prove $W(n) = O(n)$.
\end{example}

\begin{proof}
  Prove by exploiting the fact that $W(n)$ is leaf-dominated (Example 11.6).

  \begin{itemize}
  \item By definition, there exists $c_1,c_2$ such that $W(n) \le 2W(n/2) + c_1 \sqrt{n} + c_2.$
  \item Let $C(v)$ be the cost of node $v$ in the \emph{recursion tree}.  If $v$ is a node of size
    $n$, then $C(v) = c_1 \sqrt{n} + c_2$, and
    $\sum_{u \in D(v)} C(u) = 2(c_1 \sqrt{n/2} + c_2) = \sqrt{2} c_1 \sqrt{n} + 2 c_2$, where by
    $D(v)$ we mean the set of children of $v$ in the recursion tree.
  \item We have:
    \begin{align*}
      \dfrac{\sum_{u \in D(v)} C(u)}{C(v)}
      = &~ \dfrac{\sqrt{2} c_1 \sqrt{n} + 2 c_2}{c_1 \sqrt{n} + c_2} \\
      = &~ \dfrac{\sqrt{2} + (2 c_2 / c_1) / \sqrt{n}}{1 + (c_2 / c_1) / \sqrt{n}} & \mbox{(by dividing by $c_1 \sqrt{n}$)} \\
      \ge &~ \sqrt{2}.
    \end{align*}
  \item So when walking down the tree, the cost of a level in the recursion tree increases more than
    a factor of $\sqrt{2}$.
  \item Thus the leaves dominate the cost, and $W(n) = O(\mbox{the number of leaves}) = O(n)$.
  \end{itemize}
\end{proof}


\begin{example}[Induction] Suppose $W(n) = W(n/2) + O(n)$.  Prove $W(n) = O(n)$.
\end{example}

\begin{proof}
  First of all, by definition, there exists $c_1,c_2$ such that $W(n) \le W(n/2) + c_1 n + c_2$ for
  $n > 1$, and $W(1) = c_2$.  We prove $W(n) \le (2 c_1 + c_2) n + c_2$ by induction.

  \begin{itemize}
  \item Base case $n=1$: We have $W(n) = W(1) \le c_2 \le (2 c_1 + c_2) n + c_2$.

  \item Inductive case for $n$: We have:

    \begin{align*}
      W(n)
      \le &~ W(n/2) + c_1 n + c_2 & \mbox{(by assumption)} \\
      \le &~ ((2 c_1 + c_2) (n/2) + c_2) + c_1 n + c_2 & \mbox{(by inductive hypothesis)} \\
      = &~ c_1 n + (c_2 / 2) n + c_1 n + 2 c_2 \\
      = &~ (2 c_1 + c_2)n + c_2 (2 - n/2) \\
      \le &~ (2 c_1 + c_2)n + c_2. & \mbox{(since $n \ge 2$)}
    \end{align*}

  \item By induction, we have $W(n) \le (2 c_1 + c_2) n + c_2$ for all $n$.  Thus $W(n) = O(n)$.
  \end{itemize}
\end{proof}

\begin{example}[Root-dominated, advanced] Suppose $W(n) = W(\sqrt{n}) + W(n/2) + n$.  Prove
  $W(n) = O(n)$.
\end{example}


\begin{proof}
  Prove by exploiting the fact that $W(n)$ is root-dominated (page 95).

  \begin{itemize}
  \item Let $C(v)$ be the cost of node $v$ in the \emph{recursion tree}.  If $v$ is a node of size
    $n$, then $C(v) = n$, and $\sum_{u \in D(v)} C(u) = \sqrt{n} + (n/2)$, where by $D(v)$ we mean
    the set of children of $v$ in the recursion tree.
  \item We have:
    \begin{align*}
      \dfrac{\sum_{u \in D(v)} C(u)}{C(v)}
      = &~ \dfrac{\sqrt{n} + (n/2)}{n} \\
      = &~ \dfrac{1 + 2/\sqrt{n}}{2} & \mbox{(by dividing by $n/2$)} \\
      \le &~ 3/4. & \mbox{(if $2/\sqrt{n} \le 1/2$)}
    \end{align*}
  \item So if $2/\sqrt{n} \le 1/2$, or $16 \le n$, then when walking down the tree, the cost of a
    level in the recursion tree decreases more than a factor of $4/3$.
  \item Thus the root dominates the cost, and $W(n) = O(n)$.
  \end{itemize}
\end{proof}


\begin{example}[Leaf-dominated, advanced] Suppose $W(n) = W(n/2) + W(n/3) + 1$.  Prove
  $W(n) = O(n^{0.788})$.
\end{example}

\begin{proof}
  Prove by exploiting the fact that $W(n)$ is leaf-dominated (page 96).

  \begin{itemize}
  \item The recursion three is clearly leaf-dominated (exercise for the readers!).  Thus
    $W(n) = O(\mbox{the number of leaves})$.  Let's count it.

  \item Let $L(n)$ be the number of leaves when the size of the root is $n$.  Then we have
    $L(n) = L(n/2) + L(n/3)$.  Now it's clearly $L(n) = O(n)$, but we want to find a tighter bound.

  \item (Scribble: Let's guess that $L(n) = n^\beta$ for some $\beta$.  Then
    $n^\beta = (n/2)^\beta + (n/3)^\beta = n^\beta ((1/2)^\beta + (1/3)^\beta)$, and thus
    $1 = (1/2)^\beta + (1/3)^\beta$.  For this equation, there exists a real solution
    $\beta \le 0.788$.)

  \item As if we didn't scribble anything, just use the magic number $0.788$ and prove
    $W(n) = O(n^{0.788})$ by induction.
  \end{itemize}
\end{proof}

\end{document}
